\documentclass[notes.tex]{subfiles}


\begin{document}
\lecture{37}{2016--04--25}

\begin{defn}
	If $F\subseteq K$ are fields, and $\theta\in K$, let $F(\theta)$ be the smallest subfield of $K$ with $F\subseteq\theta$ and $\theta\in F(\theta)$.

	Formally, $F(\theta) = \bigcap_{\substack{L\subseteq K\text{ a field}\\F\subseteq L\\\theta\in L}} L$.
\end{defn}

\begin{quote}
	
\end{quote}

You're in $F$, some alien who sees more dimensions and stuff sees $K$. It gives you an element $\theta$ from $K$ as a gift. If $\theta$ is algebraic over $F$, then you've gained nothing \frownie.

\begin{proposition}
	Suppose $F\subseteq K$ are fields, $\theta\in K$ is algebraic over $F$. 
	Then $F(\theta)\cong F[x]/(m(x))$ where $m(x)\in F[x]$ is the minimal polynomial of $\theta$.
\end{proposition}
\begin{proof}
	Consider the map $\vp:F[x]\to K$, $\vp(p(x)) = p(\theta)$ (i.e., $\vp = \mathrm{eval}_\theta$). We know that $\Ker(vp) = (m(x))$.

	By the first isomorphism theorem, we know $\im(\vp)\cong F[x]/(m(x))$.
	It only remains to show that $\im(\vp) = F(\theta)$.
	\begin{claim}[1]
		$F(\theta)\subseteq \im(\vp)$.
		Need to check 
		\begin{enumerate}
			\item $\im(\vp)\subseteq K$ \checkmark 

			(It's a field, since it's $\cong F[x]/(m(x))$)
			\item $F\subseteq \im(\vp),$ since $\forall x\in F$
			$\vp(x) = x$
			\item $\theta\in \im(\vp)$. $\vp(x) = \theta$.
		\end{enumerate}

		Hence $F(\theta)\subseteq \im(\vp)$
	\end{claim}
	\begin{claim}[2]
		Fix $k\in \im(\vp)$ So $\exists f_0, \ldots, f_n\in F$.
		such that $k = f_n\theta^n + f_{n-1}\theta^{n-1} + \ldots + f_1\theta+f_0$. So $k\in F(\theta.)$
	\end{claim}
	\qedhere(Prop.)
\end{proof}

\begin{eg}
	Consider $\QQ\subseteq\CC$.
	Put $\theta_1 = \sqrt[3]{2}$, $\theta_1 = (\frac{-1}{2} + \frac{\sqrt{3}}{2}\iu)\sqrt[3]{2} = \sqrt[3]{2}\eu^{\iu 2\pi/3}$

	$\QQ(\theta_1)\ne \QQ(\theta_2)$ since $\QQ(\theta_1)\subset \RR$ but $\QQ(\theta_2)\nsubseteq\RR$.
	But, $\QQ(\theta_1)\cong\QQ(\theta_2)$ as fields. $\theta_1, \theta_2$ both have minimal polynomial $x^3 - 2$.
\end{eg}

\section*{Iterated Extensions}
\begin{theorem}
	Suppose $F\subseteq K\subseteq L$ are fields with $[K:F] = m < \infty$ and $[L:K] = n < \infty$.
	Then $[L:F] = [L:K][K:F] = mn$.
\end{theorem}

\begin{proof}
	In the finite case, if $|F| = q < \infty$ then $|K| = q^m$ and $|L| = |K|^n = (q^m)^n = q^{mn} = |F|^{mn}$

	Fix a basis $B = \{\beta_1,\ldots, \beta_m\}$ for $K$ over $F$ and $C = \{\gamma_1,\ldots, \gamma_n\}$ for $L$ over $K$.

	Put $A = \{\beta_i\gamma_j, : i \in[m], j\in[n]\}$

	\begin{claim}
		$A$ is a basis for $L$ over $F$.
	\end{claim}
	\begin{proof}[Pf of Claim]
		We need to show that $\forall\ell\in L,$ there is a unique sequence $(f_{i, j})_{\substack{i\in[m]\\j\in[m]}}$ of elements from $F$ such that
		\[
			\ell 
			= \sum_{\substack{i\in[m]\\j\in[n]}}f_{i, j}\beta_i \gamma_j
			= \sum_{j\in[n]}\gamma_j\underbrace{\left(\sum_{i\in[m]} f_{i, j}\beta_i \right)}_{\text{denote by }k_j\in K}
		\]
		So, $\ell = \sum_{j\in[n]}k_j \gamma_j$.

		There exists a unique $k_1, \ldots, k_n$ making this true, since $C$ is a basis for $L$ over $K$. Thus, $A$ is a basis for $L$ over $F$, as desired.\qedhere(Claim)
	\end{proof}\qedhere(Thm.)
\end{proof}

\begin{defn}
	Let $F\subseteq K$ be fields $, \theta_1, \theta_2\in K$.
	Let $F(\theta_1, \theta_2)$ be the smallest subfield of $K$ containing $F, \theta_1, \theta_2$ . So, $F(\theta_1, \theta_2) = (F(\theta_1))(\theta)_2$. 
\end{defn}
\begin{corollary}
	If $\theta_1$ and $\theta_2$ are algebraic, 
	\[
		[F(\theta_1, \theta_2):F] 
		= \underbrace{[F(\theta_1, \theta_2):F(\theta_1)]}_{\substack{\text{deg of min'l poly}\\ \text{of }\theta_2\text{ in }F(\theta_1)}} \underbrace{[F(\theta_1):F]}_{\substack{\text{deg of min'l poly}\\\text{of }\theta_1 \text{ in }F}}
	\]
\end{corollary}

\begin{eg}
	$[\QQ(\theta_1, \theta_2):\QQ] = 6$
	Note $\sqrt{3}\iu\in\QQ(\theta_1, \theta_2)$.

	$\frac{\theta_2}{\theta_1} = \frac{-1}{2} + \frac{\sqrt{3}}{2}\iu$. Has minimal polynomial $x^2+3$

	$\QQ\subseteq \QQ(\sqrt{3}\iu)\subseteq\QQ(\theta_1, \theta_2)$.
	Hence $[\QQ(\theta_1, \theta_2):\QQ]$ is even.

	To compute the 6, we need $[\QQ(\theta_1, \theta_2):\QQ(\theta_1)] = 2$

	$x^3 -2 = (x-\theta_1)(x^2+\theta_1x+\theta_1^2)$
\end{eg}

\end{document}