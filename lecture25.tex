\documentclass[notes.tex]{subfiles}

\begin{document}
\lecture{25}{2016--03--23}
\begin{center}
``When you're a professor you don't make mistakes, you make teachable moments.''
\end{center}
\begin{defn}
	If $f:X\to Y, A\subseteq X, B\subseteq Y$, $f^{-1}[B] = f^{-1}(B) = \{x\in X : f(x) \in B\}$.
\end{defn}

\begin{proposition}
	Suppose $\vp:R\to S$ is a ring homomorphism with $B\subseteq S$ then,
	\begin{enumerate}
		\item $B$ is a subring of $S\implies \inv\vp[B]$ is a subring of $R$.
		\item $B$ is an ideal of $S\implies\inv\vp[B]$ is an ideal of $R$.
	\end{enumerate}
\end{proposition}
The proof is left as an exercise.

\begin{proposition}
	If $\vp:R\to S$ is a surjective ring homomorphism, and $I\subseteq R$ is an ideal, then $\vp[I]\subseteq S$ is an ideal of $S$.
\end{proposition}
The proof is left as an exercise.

\begin{defn}
	An ideal $I\subseteq R$ is a \kw{proper ideal} if $I \ne R$.
\end{defn}

\begin{defn}
	An ideal $M\subseteq R$ is a \kw{maximal ideal} if:
	\begin{enumerate}
		\item $M$ is proper
		\item If $I$ is a proper ideal with $M\subseteq I$, then $I=M$.
	\end{enumerate}
\end{defn}

\begin{remark}[(s)]
	\begin{itemize}
		\item The zero ring $R=\{0\}$ has no proper ideals. Hence, it has no maximal ideal.
		\item If $R$ is a commutative ring with a \one, then $R$ is a field iff $\{0\}$ is maximal.
		\item Fact/Axiom If $R$ has a \one then every proper ideal is contained in a maximal ideal. (This requires set theory). (We'll never actually use this in class probably).
	\end{itemize}
\end{remark}

\begin{proposition}
	Suppose $R$ is a commutative ring with a \one and $M\subseteq R$ is an ideal. Then $M$ is maximal iff $R/M$ is a field.
\end{proposition}
\begin{proof}
	$(\Longrightarrow)$

	Take $M$ to be a maximal ideal. Since $M$ is proper, $\one\notin M$. Hence in $R/M$ we have $1+M \ne 0+M$.

	Just need to show that the only ideals of $R/M$ are $\{0+M\}$ and $R/M$.

	Suppose $I\subseteq R/M$ is an ideal. Fix $\vp:R\to R/M$ to be the ``quotient homomorphism'' $\vp(r) = r+M$.

	$\inv\vp[I]$ is an ideal in $R$. Moreover. if $m\in M$ then $\vp(m) = m+M = \zero +M \in I$. So, $M\subseteq\inv\vp[I]$. Now we can use maximality. Since $M$ is maximal, there are two options for $\inv\vp[I]$.

	If $\inv\vp[I] = M$ then $I = \{0+M\}$.

	If $\inv\vp[I] = R$ then $I = R/M$.

	Thus, $R/M$ is a field.

	$(\Longleftarrow)$ (Sketch) $R/M$ is a field and $M\subseteq I \subseteq R$, with $I$  an ideal. Look at $\vp[I] = \{\zero+M\}$ or $R/M$. (Full proof is left as an exercise).
\end{proof}

\begin{defn}
	$R$ is a commutative ring with a \one. An ideal $P\subseteq R$ is a \kw{prime ideal} if:
	\begin{enumerate}
		\item $P$ is proper
		\item $\forall a, b\in R$, if $ab\in P$ then either $a\in P$ or $b\in P$
	\end{enumerate}
\end{defn}

\begin{proposition}
	Let $R$ be a commutative ring with a \one, and $P\subseteq R$ is an ideal. Then $P$ is prime iff $R/P$ is an integral domain.
\end{proposition}
\begin{proof}[Proof (Sketch)]
	$(\Longrightarrow)$
	$P$ prime, want $R/P$ to be an integral domain.
	Toward a contradiction, suppose that $r+P, s+P\ne \zero + P$ with $rs + P = (r+P)(s+P) = 0 + P$. I.e., $rs\in P$. As $P$ is prime, one of $r, s$ is in $P$. If $r\in P$ then $r+P = \zero + P$, contradiction. By symmetry, $s\in P$ is also a contradiction.

	$(\Longleftarrow)$ Similar.
\end{proof}

\begin{corollary}
	If $R$ is a commutative ring with a \one and $I\subseteq R$ is an ideal, then $I$ is maximal $\implies$ $I$ is prime.
\end{corollary}
\begin{proof}
	If $R/I$ is a field, it's an integral domain. Then simply apply the previous proposition.
\end{proof}

\begin{eg}
	Examples in $\ZZ$:
	\begin{itemize}
		\item Every ideal of $\ZZ$ is principal.
		\item Every subgroup of $(\ZZ, +)$ has the form $n\ZZ = (n)$ for some $n\in \NN_0$.
		\item Which of the $(n)$ are prime? Answer: when $n$ is prime or zero.

		If $p$ is prime, then $z\in (p)$ iff $p\divides z$.
		So, $ab\in (p)\implies a\in (p)$ or $b\in (p)$
		If $n\ne 0$ is composite, say $n=ab$. $ab\in n, a\notin(n), b\notin(n)$

		\item Which $(n)$ are maximal?

		If $(n)$ it is prime, so $n=0$ or $n$ is a prime number.
		$(0)$ is not maximal (as $(0) \subsetneq (2)\subsetneq \ZZ$).

		If $p$ is prime, then $(p)$ is maximal. Why? Fix $a\notin (p)$. Show that $(\{p, a\}) = \ZZ$ (where $(\{p, a\})$ is the ideal generated by the set $\{p, a\}$)
		Since $\gcd(p, a) =1$ fix $x, y\in \ZZ$ such that $px + ay = 1$. $px = ay$ is in any ideal containing both $p$ and $a$. So, $\one\in (\{p, a\})$. Hence $(\{p, a\}) = \ZZ$.

		\item So in $\ZZ,$ nonzero prime ideal $\iff$ maximal ideal

		\item Bonus facts: Suppose $a, b, c\in\ZZ$ all non-zero.
		\begin{itemize}
			\item If $(c) = (a) \cap (b) \iff \lcm(a, b) = c$
			\item $(c) = (a) + (b) \iff \gcd(a, b) = c$
		\end{itemize}
	\end{itemize}
\end{eg}
\end{document}
