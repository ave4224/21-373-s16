\documentclass[notes.tex]{subfiles}

\begin{document}

\lecture{2}{2016--01--13}

\begin{definition}
	Given $a, b\in\ZZ$, denote by $\ZZ(a, b)$ the set $\{ax+by | x,y\in\ZZ\}$.
\end{definition}

\begin{theorem}[Euclid, Bezout]\index{B\'ezout's identity}
	 Suppose $a, b\in\ZZ$ are nonzero and let $d$ be the smallest positive element of $\ZZ(a, b)$, then $d$ is the unique positive GCD of $a$ and $b$.
\end{theorem}
\begin{proof}
	$d$ is a GCD of $a, b$
	\begin{enumerate}
		\item (Existence of positive GCD)
		\begin{enumerate}
			\item By integer division, $\exists q \in \ZZ, \exists r\in \ZZ$ with $0\le r < d$ such that $a = qd+r$. If $r=0$ then $d|a$, so done.
			Otherwise, suppose $0 < r < d$, so $r = a-qd$
			since $d\in\ZZ(a, b)$, we may fix $x, y$ st $d=ax+by$, meaning $r=a-q(ax+by) = a(1-qx) + b(-qy)$, so $r\in \ZZ(a, b)$, meaning $d$ was not the minimal positive element in $\ZZ(a, b)$, RAA.
			Thus, $d\divides a$
			\item
			Homework: If $c\divides a$ and $c\divides b$ then $c\divides(ax+by)$ for all $x, y\in\ZZ$ Hence $c\divides d$
		\end{enumerate}
		\item (Uniqueness of positive GCD)
			Suppose $d_1, d_2$ are both positive GCDs of $a$ and $b$.
			$d_1\divides d_2$ and $d_2 \divides d_1$ as they are both GCDs. i.e., $\exists m, n \in\ZZ$ such that $d_2=md_1$ and $d_1=nd_2$. As $\sgn(d_1) = \sgn(d_2$), $m \ge 0$ and $n \ge 0$. As $d_1 = mn d_1$, $m = n = 1$. Thus $d_1 = d_2$.
	\end{enumerate}
\end{proof}

\begin{definition}
	\kw{Relatively prime} $\iff$ $\gcd(a, b) = 1$
\end{definition}

\begin{theorem}
	If $p$ is prime, $a, b\in\ZZ$ are nonzero, and $p\divides (ab)$, then $p\divides a$ or $p\divides b$.
\end{theorem}

\begin{proof}
	Consider $d = \gcd(p, a)$. Since $d\divides p$, we know $d=p$ or $d = 1$.
	\begin{itemize}
		\item[] If $d=p$: By def of GCD, $d\divides p$ and $d\divides a$ i.e. $p\divides p$ and $p \divides a$ so we're done.
		\item[] If $d=1$: Fix integers $x$ and $y$ such that $px+ay=1$. $b=p(xb)+(ab)y$
		as $p\divides p(xb)$ and $p\divides \underbrace{(ab)}_{\uparrow}y$, $p\divides b$.
	\end{itemize}
\end{proof}

\begin{theorem} [Unique Prime Factorization]
\index{Unique Prime Factorization}
	Suppose that $a > 1$ an integer, $m, n \ge 1$ and $p_1 \le p_2 \le \ldots \le p_m,  q_1 \le q_2\le \ldots \le q_m$ are positive primes.
\end{theorem}

Then $m=n$ and $p_i=q_i$ for all $i$.
\begin{proof}
	By induction, it suffices to show $p_1 = q_1$. Suppose not.
WLOG, assume $p_1 < q_1$. We know that $p_1\divides a$ (as $p_1\divides q_1q_2\ldots q_n$) Hence, $\exists i\le n$ such that $p_i \divides q_i$. since $p_i$ and $q_i$ prime, $p_1 = q_i$. However, $p_1 < q_1 \le q_i = p_1$ so $p_1 < p_2$ contradiction.

Hence $p_1 = q_1$ so by induction, we're done.
\end{proof}

\end{document}
