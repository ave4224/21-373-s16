\documentclass[notes.tex]{subfiles}

\begin{document}

\lecture{12}{2016--02--10}

Note: group actions can be either left actions or right actions. However, we will only talk about left actions in this course, so we will refer to them exclusively as ``actions.''


Alternate def $G\to S_X$.

Important special case: $X = G$, then $G\actson G$ ($G$ acts on itself).

There are three main actions:
\begin{itemize}
	\item $G\actson G$ by left multiplication.
	$\forall g\in G, x\in X (=G), g\cdot x = gx$.
	\item $G\actson G$ by right multiplication.
	$g, x\in G, g\cdot x = x\inv g$.
	\item $G\actson G$ by conjugation.
	$g\cdot x = g x \inv g$.
	Note that $gx\inv g$ is simply $x$ conjugated by $\inv g$, so it doesn't matter whether we write $\inv gxg$ or $gx\inv g$.
\end{itemize}

\begin{theorem}[Cayley]\index{Cayley's Theorem}
	Suppose $G$ is a group. Then there is a set $X$ and a subgroup $H\le S_X$ such that $G\cong H$.

	Moreover, we can choose $X$ to have cardinality $|G|$. (i.e., if $|G| = n$, we can find an isomorphic copy of $G$ inside $S_n$.)
\end{theorem}


\begin{proof}
	Take $X = G$ and consider the action $G\actson G$ by left multiplication ($g\cdot x = gx$).

	This induces a homomorphism $\vp:G\to S_G, \vp(g) = \lambda_g$ where $\lambda_g(x) = gx$.
	\begin{claim}
		$\Ker(\vp) = \{e_G\}$.
	\end{claim}
	\begin{proof}[Proof (Claim):]
	Suppose $g\in G$ such that $\vp(g) = e \in S_G$. so $\lambda_g = e$.

	In particular, $e_G = \lambda_g(e_G) = ge_G = g$, so $g = e_G$.\qedhere(Claim)
	\end{proof}
	By the first isomorphism theorem, $G/\Ker(\vp) \cong \im(\vp)$. Denote by $H$ the image of $\vp$. $H \le S_G$.

	$G/\Ker(\vp) = G/\{e_G\} \cong G$, so $G\cong H$. \qedhere(Theorem)
\end{proof}

\begin{eg}
	A concrete example:

	$G = \Zn2\times\Zn2 = \{1, a, b, c\}$

	% TODO: INSERT MULTIPLICATION TABLES HERE

	$\lambda_b: \begin{cases}
		1\mapsto 3\\
		2\mapsto 4\\
		3\mapsto 1\\
		4\mapsto 2
	\end{cases}$
	Run cycle decomposition on each % to get
	% TODO: INSERT CYCLE DECOMPOSITIONS HERE

	$G\cong \{e, \cyc{1 2}\cyc{3 4}, \cyc{1 3}\cyc{2 4}, \cyc{1 4}, \cyc{2 3}\} \le S_4$.

	This is sometimes called the left multiplication permutation representation of a group.
\end{eg}

\begin{remark}
	Cayley's theorem is not always ``optimal.'' Sometimes $|G| = n$ and $m < n$ such that $H\le S_m$ and $G\cong H$.
\end{remark}

\begin{eg}
	$\Zn6 = G, |G|=6$.
	Take $\sigma = \cyc{1 2 3}\cyc{4 5} \in S_5$.

	Then $H = \csg\sigma \cong \Zn6 = G$, but $H \le S_5$ and $5 < 6$.
\end{eg}

\chapter*{Orbit Equivalence Relations}

\begin{defn}
	Suppose $G\actson X$. Define a relation $\sim$ on $X$ by $x\sim y$ iff $\exists g\in G: g\cdot x = y$.

	This $\sim$ is called the \kw{orbit equivalence relation}.
\end{defn}

\begin{proposition}
	$\sim$ is an equivalence relation.
\end{proposition}

\begin{proof} 3 properties:
	\begin{itemize}
		\item Reflexivity:

		$x\in X$. $e_G\cdot x = x$, so $x\sim x$
		\item Symmetry:

		Suppose $x\sim y$.
		Fix $g\in G$ such that $g\cdot x = y$. Compute $\inv g \cdot y = \inv g \cdot(g\cdot x) = (g\inv g)\cdot x = e_G\cdot x = x$. So $y\sim x$
		\item Transitivity:

		Suppose $x\sim y, y\sim z$.
		Fix $g, h\in G$ such that $g\cdot x = y$ and $h\cdot y = z$.
		So $(hg)\cdot x = h\cdot(g\cdot x) = h\cdot y = z$, so $x\sim z$
	\end{itemize}

	It follows that $\sim$ is an equivalence relation.
\end{proof}

\begin{defn}
	The equivalence classes of $\sim$ are called \kw{orbits}.
	Write them like $\orbit_x$. Because $\sim$ is an equivalence class, $\{\orbit_x\}_{x\in X}$ partitions $X$.
\end{defn}

% TODO: INSERT PARTITION DIAGRAM?

\begin{notation}
	Sometimes we write $G\cdot x$ to denote the orbit of $x$.
\end{notation}
\end{document}
