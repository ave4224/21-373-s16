%\ProvidesPackage{jvb}
%
% Author: Jacob Van Buren
% Last Update: 2013-07-17
% Version: v1.0.2
% Description:
%     Collection of helpful formatting macros and commands
%     Suitable for TeXing algorithms homework
%

%%%%%%%%%%%%%%%%%%%%% Imports %%%%%%%%%%%%%%%%%%%%%
\RequirePackage[utf8]{inputenc}
% \RequirePackage{fixltx2e}
\RequirePackage[T1]{fontenc}
%%\RequirePackage[margin=0.5in,top=1in]{geometry}
%\RequirePackage{algorithm}
%\RequirePackage{amsfonts}
%\RequirePackage{amssymb}
\usepackage{amsmath}
\usepackage{amssymb}
\usepackage{amsthm}
\usepackage{cooltooltips}
\usepackage{fancyhdr}
\usepackage{graphicx}
\usepackage{lastpage}
\usepackage{mathtools}
\usepackage[nodisplayskipstretch]{setspace}
\usepackage{wrapfig}
\usepackage{multicol}
\usepackage{changepage}
% \usepackage{titlesec}

\usepackage{mathrsfs}

%\usepackage{xstring}

%\setlength{\headheight}{15.2pt}

% Macros, taken from COS 423 template
\DeclareMathOperator{\BigOm}{\mathcal{O}}
\newcommand{\BigOh}[1]{\BigOm\left({#1}\right)}
\DeclareMathOperator{\BigTm}{\Theta}
\newcommand{\BigTheta}[1]{\BigTm\left({#1}\right)}
\DeclareMathOperator{\BigWm}{\Omega}
\newcommand{\BigOmega}[1]{\BigWm\left({#1}\right)}
\DeclareMathOperator{\LittleOm}{\mathrm{o}}
\newcommand{\LittleOh}[1]{\LittleOm\left({#1}\right)}
\DeclareMathOperator{\LittleWm}{\omega}
\newcommand{\LittleOmega}[1]{\LittleWm\left({#1}\right)}


\newcommand{\calP}{\mathcal{P}}
\newcommand{\Z}{\mathbb{Z}}
\newcommand{\R}{\mathbb{R}}
\newcommand{\Exp}[1]{\mathbb{E}\left[#1\right]}
\newcommand{\Q}{\mathbb{Q}}
\newcommand{\sign}{\mathrm{sign\ }}
\newcommand{\abs}{\mathrm{abs\ }}
\newcommand{\eps}{\varepsilon}
\newcommand{\zo}{\{0, 1\}}
\newcommand{\SAT}{\mathit{SAT}}
\renewcommand{\P}{\mathbf{P}}
\newcommand{\NP}{\mathbf{NP}}
\newcommand{\coNP}{\co{NP}}
\newcommand{\co}[1]{\mathbf{co#1}}
\newcommand{\Pri}[2][]{\ensuremath{\mathop{\mathbf{Pr}}\nolimits_{#1}\left[{#2}\right]}}
\renewcommand{\Pr}[2][]{\ensuremath{\mathop{\mathbf{Pr}}\limits_{#1}\left[{#2}\right]}}
\newcommand{\DTIME}[1]{\mathbf{DTIME}\left[#1\right]}
\newcommand{\POLYLOG}[1]{\mathrm{POLYLOG}\left(#1\right)}
\newcommand{\OPT}{\ensuremath{\text{OPT}}}
\newcommand{\suchthat}{\ensuremath{\text{ s.t. }}}
\newcommand{\xor}{\oplus}
\DeclareMathOperator{\sgn}{\mathrm{sgn}}

\newlength{\thmspacing}
\setlength{\thmspacing}{8.0pt plus 2.0pt minus 4.0pt}
\newtheoremstyle{boldplain}
  {\thmspacing}   % ABOVESPACE
  {\thmspacing}   % BELOWSPACE
  {\itshape}  % BODYFONT
  {0pt}       % INDENT (empty value is the same as 0pt)
  {\bfseries} % HEADFONT
  {.}         % HEADPUNCT
  {5pt plus 1pt minus 1pt} % HEADSPACE
  {}          % CUSTOM-HEAD-SPEC

\newtheoremstyle{bolddef}
  {\thmspacing}   % ABOVESPACE
  {\thmspacing}   % BELOWSPACE
  {\normalfont}  % BODYFONT
  {0pt}       % INDENT (empty value is the same as 0pt)
  {\bfseries} % HEADFONT
  {.}         % HEADPUNCT
  {5pt plus 1pt minus 1pt} % HEADSPACE
  {}          % CUSTOM-HEAD-SPEC


\newtheoremstyle{boldremark}
  {0.5\thmspacing}   % ABOVESPACE
  {0.5\thmspacing}   % BELOWSPACE
  {\normalfont}  % BODYFONT
  {0pt}       % INDENT (empty value is the same as 0pt)
  {\itshape} % HEADFONT
  {.}         % HEADPUNCT
  {5pt plus 1pt minus 1pt} % HEADSPACE
  {}          % CUSTOM-HEAD-SPEC

\makeatletter
\renewenvironment{proof}[1][\proofname] {\par\pushQED{\qed}\normalfont\topsep6\p@\@plus6\p@\relax\trivlist\item[\hskip\labelsep\itshape#1\@addpunct{.}]\ignorespaces}{\popQED\endtrivlist\@endpefalse}
\makeatother

\theoremstyle{boldplain}
\newtheorem{theorem}{Theorem}[chapter]
\newtheorem*{theorem*}{Theorem}
\newtheorem{thm}[theorem]{Theorem}
\newtheorem{lemma}{Lemma}[theorem]
\newtheorem{invariant}[theorem]{Invariant}
\newtheorem{corollary}{Corollary}[theorem]
\newtheorem{proposition}[theorem]{Proposition}


\theoremstyle{bolddef}
\newtheorem{definition}[theorem]{Definition}
\newtheorem{defn}[theorem]{Definition}
\newtheorem{property}[theorem]{Property}

\theoremstyle{boldremark}
\newtheorem*{claim}{Claim}
\newtheorem*{notation}{Notation}
\newtheorem*{remark}{Remark}
\newtheorem*{eg}{Example}
\newtheorem*{exercise}{Exercise}

% My Macros


%sets
\def \setN {\mathbb{N}}
\def \setQ {\mathbb{Q}}
\def \setR {\mathbb{R}}
\def \setZ {\mathbb{Z}}
\def \setC {\mathbb{C}}
\def \NN {\mathbb{N}}
\def \QQ {\mathbb{Q}}
\def \RR {\mathbb{R}}
\def \ZZ {\mathbb{Z}}
\def \CC {\mathbb{C}}
\def \vs {\emptyset}
\def \pset {\mathbb{P}}
% \def \begin{tabin} {\begin{adjustwidth}{1cm}{0pt}}
% \def \end{tabin} {\end{adjustwidth}}
\def \divides {\ensuremath{\mkern2mu|\mkern1mu}}
\def \ndivides {\ensuremath{\mkern1mu{\not{|}}\mkern3mu}}

%linear algebra (TeX macros)

\def\matthree #1,#2,#3|#4,#5,#6|#7,#8,#9 {\left(\begin{array}{ccc}#1&#2&#3\\#4&#5&#6\\#7&#8&#9\end{array}\right)}

\def\mattwo #1,#2|#3,#4 {\left(\begin{array}{cc}#1&#2\\#3&#4\end{array}\right)}
\def\mattwothree #1,#2,#3|#4,#5,#6 {\left(\begin{array}{ccc}#1&#2&#3\\#4&#5&#6\end{array}\right)}
\def\mattwofour #1,#2,#3,#4|#5,#6,#7,#8 {\left(\begin{array}{cccc}#1&#2&#3&#4\\#5&#6&#7&#8\end{array}\right)}
\def\matfourtwo #1,#2|#3,#4|#5,#6|#7,#8 {\left(\begin{array}{cc}#1&#2\\#3&#4\\#5&#6\\#7&#8\end{array}\right)}
\def\matthreetwo #1,#2|#3,#4|#5,#6 {\left(\begin{array}{cc}#1&#2\\#3&#4\\#5&#6\end{array}\right)}


\def\smattwo #1,#2|#3,#4 {\left(\begin{smallmatrix}#1&#2\\#3&#4\end{smallmatrix}\right)}
\def\smattwothree #1,#2,#3|#4,#5,#6 {\left(\begin{smallmatrix}#1&#2&#3\\#4&#5&#6\end{smallmatrix}\right)}
\def\smattwofour #1,#2,#3,#4|#5,#6,#7,#8 {\left(\begin{smallmatrix}#1&#2&#3&#4\\#5&#6&#7&#8\end{smallmatrix}\right)}
\def\smatthreetwo #1,#2|#3,#4|#5,#6 {\left(\begin{smallmatrix}#1&#2\\#3&#4\\#5&#6\end{smallmatrix}\right)}

\def\pmattwo #1,#2|#3,#4 {\left(\begin{array}{cc}#1&#2\\#3&#4\end{array}\right)}

\def\matthreefracs #1,#2,#3|#4,#5,#6|#7,#8,#9 {\left(\begin{array}{ccc}#1&#2&#3\\#4&#5&#6\\[3pt]#7&#8&#9\end{array}\right)}

%vertical
\def\vectwo #1,#2               {\left(\begin{matrix}#1\\#2\end{matrix}\right)}
\def\vectwosmall #1,#2          {\left(\begin{matrix}#1&#2\end{matrix}\right)}
\def\vecthree #1,#2,#3          {\left(\begin{matrix}#1\\#2\\#3\end{matrix}\right)}
\def\vecfour #1,#2,#3,#4        {\left(\begin{array}{c}#1\\#2\\#3\\#4\end{array}\right)}
\def\vecfourrow #1,#2,#3,#4     {\left(\begin{matrix}#1&#2&#3&#4\\\end{matrix}\right)}
\def\vecfive #1,#2,#3,#4,#5     {\left(\begin{matrix}#1\\#2\\#3\\#4\\#5\end{matrix}\right)}
\def\vecsix #1,#2,#3,#4,#5,#6   {\left(\begin{matrix}#1\\#2\\#3\\#4\\#5\\#6\end{matrix}\right)}

%horizontal
\def\hvectwo #1,#2              {\left\langle #1,#2\right\rangle}
\def\hvecthree #1,#2,#3         {\left\langle #1,#2,#3\right\rangle}
\def\hvecfour #1,#2,#3,#4       {\left\langle #1,#2,#3,#4\right\rangle}
\def\hvecfive #1,#2,#3,#4,#5    {\left\langle #1,#2,#3,#4,#5\right\rangle}
\def\hvecsix #1,#2,#3,#4,#5,#6  {\left\langle #1,#2,#3,#4,#5,#6\right\rangle}


\def\ij{\mkern1mui\mkern1mu}
\def\mstart{\left(\begin{array}}%{c c ...etc... c}
\def\mend{\end{array}\right)}
\def\smstart{\left(\begin{smallmatrix}}
\def\smend{\end{smallmatrix}\right)}
\def\vstt{\left(\begin{array}{c}}
\def\vend{\end{array}\right)}
\def\becomes{\;\raise.17ex\hbox{\ensuremath{\scriptstyle\sim}}\;}
\def\stbox{\mkern2mu\rule[.2ex]{2pt}{2pt}\mkern2mu}
\def\RAA{\ensuremath{\mathit{R.A.A.}}}
\def\Span#1{\ensuremath{\mathrm{Span}\left(#1\right)}}
\newcommand\closure{\ensuremath{\mathscr{C}}}
\newcommand{\txtif}{\ensuremath{\text{if }}}
\newcommand{\txtow}{\ensuremath{\text{otherwise}}}
\newcommand{\txtelse}{\ensuremath{\text{else }}}

\def\tr#1{{#1}^\mathrm{T}}
\def\inv#1{{#1}^{-1}}
\newenvironment{tabin}[1][1cm]{\begin{adjustwidth}{#1}{0pt}}{\end{adjustwidth}}
% \def \begin{tabin} {\begin{adjustwidth}{1cm}{0pt}}
% \def \end{tabin} {\end{adjustwidth}}


\DeclarePairedDelimiter{\ceil}{\lceil}{\rceil}
\DeclarePairedDelimiter{\floor}{\lfloor}{\rfloor}
\DeclarePairedDelimiter{\len}{|}{|}
\DeclarePairedDelimiter{\set}{\{}{\}}
\DeclarePairedDelimiter{\paren}{(}{)}
\DeclarePairedDelimiter{\brackets}{[}{]}


\newcommand{\setnextsection}[1]{\setcounter{section}{#1}\addtocounter{-1}}
\newcommand{\skipsection}{\addtocounter{section}{1}\setcounter{subsection}{0}}
\newcommand{\setnextsubsection}[1]{\setcounter{subsection}{#1}\addtocounter{-1}}
\newcommand{\setnextsubsubsection}[1]{\setcounter{subsubsection}{#1}\addtocounter{-1}}
\renewcommand{\thesubsubsection}{\thesubsection\,(\alph{subsubsection})}


\def\noalignspacing{%
    \setlength{\abovedisplayskip}{0pt}%
    \setlength{\belowdisplayskip}{0pt}%
    \setlength{\abovedisplayshortskip}{0pt}%
    \setlength{\belowdisplayshortskip}{0pt}%
}
% \newenvironment{smallign*}{%
%     \bgroup%
%     \setlength{\abovedisplayskip}{0pt}%
%     \setlength{\belowdisplayskip}{0pt}%
%     \setlength{\abovedisplayshortskip}{0pt}%
%     \setlength{\belowdisplayshortskip}{0pt}%
%     \begin{align*}%
% }{%
%     \end{align*}%
%     \egroup%
% }

\newcommand{\mcaption}[3][1 1 0]{\cooltooltip[#1]{Subject}{#3}{}{}{#2}}
%makes mouseover captions
%Usage: \mcaption{text}{caption to display}


%%%%%%%%%%%%%%%%%% Header Macros %%%%%%%%%%%%%%%%%
\newcommand{\myname}{redefine $\backslash$myname}
\newcommand{\hworganization}{redefine $\backslash$hworganization}
\newcommand{\hwtitle}{redefine $\backslash$hwtitle}
\newcommand{\hwnum}{redefine $\backslash$hwnum}
\newcommand{\assigndate}{redefine $\backslash$assigndate}
\newcommand{\duedate}{redefine $\backslash$duedate}
\newcommand{\mysection}{redefine $\backslash$mysection}

% Header styles
\fancypagestyle{algshw}{
 \fancyhf{}
 \fancyhead[L]{
  \myname\\
  {\hworganization}
 }
 \fancyhead[C]{
  \textbf{
   \hwtitle
  }\\
  \textbf{
   \hwnum
  }
 }
 \fancyhead[R]{
  Given: \assigndate\\
  Due: \duedate
 } 
 \fancyfoot[C]{
  Page \thepage\ of \pageref{LastPage}
 }
 \renewcommand{\headrulewidth}{0.6pt}
 \renewcommand{\footrulewidth}{0.4pt}
}

\fancypagestyle{cmuhw}{
 \fancyhf{}
 \fancyhead[L]{
  \myname\\
  {\hworganization}
 }
 \fancyhead[C]{
  \textbf{
   \hwtitle
  }\\
  \textbf{
   \hwnum
  }
 }
 \fancyhead[R]{
  Section: \mysection\\
  Due: \duedate
 }
 \fancyfoot[C]{
  Page \thepage\ of \pageref{LastPage}
 }
 \renewcommand{\headrulewidth}{0.6pt}
 \renewcommand{\footrulewidth}{0.4pt}
}


