\documentclass[notes.tex]{subfiles}

\begin{document}
\lecture{27}{2016--03--28}
\chapter{Polynomials Over Integral Domains}
First, we're going to focus on polynomials over fields.
Typical polynomial in $R[x]$ looks like $p(x) = \sum_k^na_kx^k$ with $a_k\in R$ and $n\in \NN_0$.

Questions: Can we factor them? Uniquely? Can we ``solve'' polynomial equations, etc.

\begin{defn}\leavevmode
	\begin{itemize}
		\item Each $a_k$ is a \kw{coefficient}
		\item Each $a_kx^k$ is a \kw{term}
		\item The largest $k$ with $a_k\ne0$ (if it exists) is called the \kw{degree} of $p(x)$ (denoted $\deg p$)
		\item If all $a_k = 0$ we leave the degree undefined (but we can consider it to be $-\infty$ for various reasons that ``we're not going to get into'')
		\item If $\deg p = n$ we call $a_n, a_nx^n$ the \kw{leading coefficient} and the \kw{leading term}, respectively.
		\item If the leading coefficient is 1, we say $p$ is \kw{monic}.
	\end{itemize}
\end{defn}

\begin{proposition}
	Let $R$ be an integral domain, $p(x), q(x)\in R[x]$ both non-zero. Then, $\deg(p(x)q(x)) = \deg p + \deg q$.
\end{proposition}
\begin{proof}
	Suppose $\deg p = m, \deg q = n$. Apply the definition of multiplication.
	The leading term of the products is therefore $a_mb_nx^{m+n}$. ($a_mb_n\ne \zero$ because $R$ is an integral domain).
\end{proof}

\begin{corollary}
	$R$ is an integral domain.
	\begin{enumerate}
		\item Units of $R[x]$ are exactly the units of $R$ (viewed as a degree 0 polynomial.)
		\item $R[x]$ is an ID.
	\end{enumerate}
\end{corollary}
\begin{proof}\leavevmode
	\begin{enumerate}
		\item Suppose $p$ is a unit. Fix $q\in R[x]$ such that $p(x)q(x) = \one$. $\deg(\one) = 0$. So, $\deg(pq) = \deg(p) + \deg(q) = 0$
		Hence, $\deg p = \deg q = 0.$ Thus, $p(x) = a_0, q(x) = b_0$. $a_0b_0 = \one$, so $a_0, b_0$ are units in $R$.
		\item Analogous.
	\end{enumerate}
\end{proof}

\begin{theorem}[Division Algorithm for Polynomials over fields]
	\label{thm:polydivalg}
	Suppose $F$ is a field, $a(x), b(x)\in F[x]$ with $b(x)\ne 0$. Then, there are $q(x), r(x) \in F[x]$ such that
	\begin{itemize}
		\item $a(x) = q(x)b(x) + r(x)$
		\item $r = \zero$ or $\deg r < \deg b$
	\end{itemize}
	(Also $q$ and $r$ are unique, proof left as an exercise.)
\end{theorem}
\begin{proof}
	We proceed by induction on $m = \deg a$.

	If $a=\zero,$ then $q=r=\zero$.

	Suppose the theorem is true for all $a'$ (with the same $b$) with $\deg(a') < m$.

	Case 1 ($\deg a < \deg b$):
	\begin{tabin}
		$a = \zero b + a$
	\end{tabin}
	Case 2 ($\deg a \ge \deg b$):
	\begin{tabin}
		Denote by $n$ the degree of $b$.
		Fix leading terms $a_mx^m$ of $a(x)$ and $b_nx^n$ of $b(x)$.

		Define $a'(x) = a(x) - \frac{a_m}{b_n}x^{m-n}b(x)$.

		\begin{claim}
			$\deg a' < \deg a$
			(or $a'=\zero$)
		\end{claim}
		\begin{proof}[Proof (Claim)]
			Coefficients above $m$ are $\zero$. What about the coefficient of $x^m$? It equals $a_m - \frac{a_m}{b_n}b_n = \zero$.

			So, $\deg(a') < n$ (or $a' = \zero$)
		\end{proof}
		So, by induction, $\exists q', r'\in F[x]$ such that
		\begin{itemize}
			\item $a'(x) = q'(x)b(x) + r'(x)$
			\item $r'=\zero$ or $\deg r' < \deg b$
		\end{itemize}
		We know \begin{align*}
			a(x)&=a'(x) + \frac{a_m}{b_n}x^{m-n}b(x)\\
			&= q'(x)b(x) + r'(x) + \frac{a_m}{b_n}x^{m-n}b(x)\\
			&= \underbrace{\big(q'(x) + \frac{a_m}{b_n}x^{m-n}\big)}_{q(x)}b(x) + r'(x)
		\end{align*}.
	\end{tabin}
\end{proof}

We needed $F$ to be a field. Consider $a(x) = 3x+1, b(x) = 2$ in $\ZZ[x]$.

\begin{defn}
	An ID $R$ is a \kw{Euclidean domain} (\kw{ED}) if there is a function $N:R\to \NN$ such that $\forall a, b\in R, b\ne \zero$ $\exists q , r\in R $ with
	\begin{itemize}
    \item $a=qb+r$
    \item $r=0$ or $N(r) < N(b)$
  \end{itemize}
\end{defn}

\begin{notation}
	$N$ (as described above) is called a $\kw{norm}$.
	\begin{itemize}
		\item $q$ is the \kw{quotient}
		\item $r$ is the \kw{remainder}
	\end{itemize}
\end{notation}

\begin{eg}\leavevmode
	\begin{itemize}
		\item $R = \ZZ$, $N(z) = \abs z$
		\item $F[x]$ with $F$ a field, $N(p) = \deg p$
		\item $F$ a field, $N$ any function.
	\end{itemize}
\end{eg}

\begin{defn}
	An ID $R$, is a \kw{principal ideal domain} (\kw{PID}) if every ideal is principal (i.e., $I\subseteq R$ an ideal $\implies \exists a\in R (I = (a) = aR)$)
\end{defn}

\begin{proposition}
	Every Euclidean domain is a principal ideal domain.
\end{proposition}
\begin{proof}
	Fix ED $R$, $I\subseteq R$ an ideal. If $I = \{\zero\}$ then $I = (\zero)$. If $I \ne \{\zero\}$, fix $a := \arg\min_{a\in I\setminus\{\zero\}} N(a)$.
	\begin{claim}
		$I=(a)$
	\end{claim}
	\begin{proof}[Proof of Claim]
		$(a)\subseteq I$, need $I\subseteq (a)$ Fix $s\in I$. By ED, $\exists q, r\in R$ such that $s=qa+r$ with $N(r) < N(a)$ or $r=0$. $r = s-qa\in I$. Thus, as $a$ was minimal, $r=0$. Thus $s=qa \in (a)$
	\end{proof}
\end{proof}
\end{document}
