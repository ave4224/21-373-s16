\documentclass[notes.tex]{subfiles}

\begin{document}

\lecture{4}{2016--01--20}

Recall the definition of a group.
\begin{defn}
	$(G, *)$ is an \kw{abelian} (commutative) group if it is a group and
\end{defn}
\begin{enumerate}
	\item[iv.] $(G, *)$ is commutative ($\forall x, y\in G: x*y = y*x$)
\end{enumerate}

Let $(G, *)$ be an arbitrary but fixed group.

\begin{proposition}
	There is a unique identity element.
\end{proposition}

\begin{proof}
	Suppose $e$ and $f$ both satisfy the second group property.
	we compute $e*f$ in two ways. $e*f = f$ and $e*f = e$, so by transitivity, $e = f$.
\end{proof}

\begin{proposition}
	If $a\in G$, $a$ has a unique inverse.
\end{proposition}

\begin{proof}
	Suppose that $b$ and $c$ are both inverses for $a$, $b*a = e$, $a*c = e$.
	Then, \[
		b = b*e = b*(a*c) = (b*a) * c = e * c = c
	\]
\end{proof}

\subparagraph{Notational Conventions} % (fold)
\label{subp:notational_conventions}
\begin{itemize}
	\item We will often just call a group $G$ instead of $(G, *)$
	\item We abbreviate multiplication ($x*y$) as $x\cdot y$  or just $xy$
	\item We will often write $xyz$ for $(x * y) * z$ (due to associativity)
	\item When working with $(\ZZ, +)$, we'll just use $+$
	\item We'll denote the (unique) identity of $G$ by $1$ or by $e$.
	\item We'll denote the inverse of $x$ by $\inv x$
	\item Given an integer exponent $n\in\ZZ$ and $x\in G$, define 
	\[
		x^n = \begin{cases}
			\prod_{i=1}^{n}x, &\text{ if } n > 0\\
			e, &\text{ if } n = 0\\
			\prod_{i=1}^{-n}(\inv x), &\text{ if } n < 0\\
		\end{cases}
	\]
\end{itemize}

\subparagraph{Group Examples} % (fold)	
\label{subp:group_examples}
``Definition:'' $\QQ = \left\{\frac{a}{b} : a\in\ZZ, b\in\ZZ^+\right\}$

\begin{enumerate}
	\item $(\ZZ, +)$ is an abelian group
	\item $(\ZZ, \times)$ is not a group

		Why? 2 has no inverse in $\ZZ.$
		($\nexists x\in\ZZ : (2x = 1)$)
	\item $(\QQ, +)$ is an abelian group
	\item $(\QQ, \times)$ is not a group ($0$ has no inverse)
	\item $(\QQ \setminus \{0\}, \times)$ is an abelian group.
	\item $\GL(n)$ is the set of  matricies $A_{n\times n}$ for which $\det A_{n\times n} \ne 0$
	\item The set $G$ of $2\times 2$ matrices with determinant 1, along with matrix multiplication, is a group.
		Called the ``special linear group.''
		\begin{itemize}
			\item[Closure:]
			\[
				\det \left(\smattwo a,b|c,d \cdot\smattwo e,f|g,h \right) = (b c-a d) (f g-e h) = 1
			\]
			\item[i.] Associativity: proof left for the reader.
			\item[ii.] Identity: $\smattwo 1,0|0,1 $
			\item[iii.] Given $a = \smattwo a,b|c,d $, take $a^{-1} = \smattwo d,-b|-c,a $, which you can verify is still in $G$.
		\end{itemize}
		The group is \emph{not} abelian.
		Take $a = \smattwo 1,1|0,1 , b = \smattwo 1,0|1,1 $. Verify that $ab \ne ba$
	\item Suppose that $X\ne\vs$ is some set, and denote by $S_X$ the set of bijections $f:X\to X$.
		Then $(S_X, \circ)$ is a group, where $\circ$ is function composition. ($(f\circ g)$ is the function $x\mapsto f(g(x))$.)

		Identity is $x\mapsto x$. Inversion $\inv f = \inv f$.
\end{enumerate}

% subparagraph group_examples (end)
% subparagraph notational_conventions (end)
\end{document}
