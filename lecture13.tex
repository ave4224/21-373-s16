\documentclass[notes.tex]{subfiles}

\begin{document}

\lecture{13}{2016--02--12}

\begin{defn}
	$G\actson X$, fix $x\in X$. The \kw{stabilizer} of $x$ is $G_x = \{g\in G: g\cdot x = x\}\subseteq G$.
\end{defn}

\begin{proposition}
	If $G$ is a group then $G_x \le G$.
\end{proposition}

\begin{proof}
	Homework Question
\end{proof}


\begin{eg} Let's look at some examples of group actions and stabilizers of some the elements of the sets they act on.
	\begin{enumerate}
		\item
			$\sigma \in S_5$, say $\sigma = \cyc{1 3 4}\cyc{2 5}$.
			$S_5\actson\{1,2,3,4,5\}$.
			This induces an action of $\csg\sigma\actson \{1,2,3,4,5\}$. (Note that $|\sigma| = 6.$)

			$G = \csg\sigma = \{e, \sigma, \ldots, \sigma^5\}$.
			$X = \{1,2,3,4,5\}$.
			Look at $x=3$.
			$\mathcal{O} = \{1,3 4\} = $ the cycle containing $3$ in the cycle decomposition of $\sigma$.

			If we check each exhaustively, we find $e\cdot 3 = 3$ and $\sigma^3 = 3$ so the stabilizer of $G_3$ is $\{e, \sigma^3\}$.

			In general, if you have any perm group, the orbit of an element of the cyclic subgroup generated by an element in the permutation group is going to be the cycles and the stabilizers are going to be the lengths of the cycles.
			% TODO: EXPLAIN MORE
		\item
			$G\actson G$ by left multiplication.
			$g\cdot x = gx$
			\begin{claim}
				$\forall x\in G, $ $\orbit_x = G$ (its orbit is $G$).
			\end{claim}
			\begin{proof}
				Fix $x\in G,$ Fix $g\in G$. Choose $h = g\inv x$, then $h\cdot x = (g\inv x)\cdot x = (g\inv x)x = g(\inv xx) =g.$ Thus, $g\in \orbit_x$. Thus, $\orbit_x = G$.
			\end{proof}
			\begin{claim}
				$\forall x\in G, G_x = \{e_G\}$.
			\end{claim}
			\begin{proof}
				Fix $x$. Suppose $g\in G$.
				$g\cdot x = x$.

				Thus, $gx\inv x = x\inv x$ (as $X = G$), so $g = e_G$.
			\end{proof}
		\item
			$S_3\actson S_3$ by conjugation. $g\cdot x = gx\inv g$.

			\emph{Orbits} $\{e\}, \{\cyc{1 2}, \cyc{1 3}, \cyc{2 3}\}, \{\cyc{1 2 3}, \cyc{1 3 2}\}$

			The stabilizer of (1 2) is $\{e, \cyc{1 2}\}$.
		\item
			$H\le G$. Let $H\actson G$ by right multiplication.
			Then
			$$\orbit_{x\in G} = \{h\cdot x : h\in H\} = \{x\inv h: h\in H\} = \{xh : h\in H\} = xH.$$
			Thus, the orbits of the elements of $G$ are the left cosets of $H$ in $G$.
	\end{enumerate}
\end{eg}

\begin{defn}
	Suppose $G\actson X$ with a single orbit $\orbit = X.$

	We say that the element is \kw{transitive}
\end{defn}

\begin{defn}
	Orbits of the conjugation action of $G\actson G$ are called \kw{conjugacy classes}.
\end{defn}

\begin{theorem}[Orbit-Stabilizer Theorem]
  \index{Orbit-Stabilizer Theorem}\label{OST}
	Suppose $G$ is a finite group, $X$ is some set, and $G\actson X$. Fix arbitrarily $x\in X$ with orbit $\orbit \subseteq X$ and stabilizer $G_x \le G.$ Then $|\orbit|\cdot|G_x| = |G|$.
\end{theorem}

\begin{proof}
	Define two equivalence relations on $G$.
	\begin{enumerate}
		\item $g\sim h$ iff $\inv gh \in G_x$ (left coset euqivalence of stabilizers)
		\item $g\approx g$ iff $g\cdot x = h\cdot x\in X$
	\end{enumerate}
	For the reader: check $\sim$ and $\approx$ are equivalence relations.

	\begin{claim}[1]
		Each $\sim$ equivalence class has $|G_x|$ many elements in it.
	\end{claim}
	\begin{proof}
		Already done\qedhere(Claim 1)
	\end{proof}
	\begin{claim}[2]
		There are exactly $|\orbit|$-many $\approx$ equivalence classes.
	\end{claim}
	\begin{proof}
		For each $y\in \orbit$, put $A_y = \{g\in G: g\cdot x = y\}$.
		The collection of $\{A_y : y\in\orbit\}$ is exactly  the set of $\approx$-equivalent classes.

		In other words, $\approx$ is partitioning $G$ by the elements which move $x$ into each particular element of $\orbit$.
		\qedhere(Claim 2)
	\end{proof}
	\begin{claim}[3]
		$\sim \cong \approx$
	\end{claim}
	\begin{proof}
		First show $g\sim h \implies g\approx h$, then show $g\approx h \implies g\sim h$.
		\begin{itemize}
			\item Suppose $\inv gh\in G_x$, then
				$(\inv gh)\cdot x = x$, so
				$\inv g \cdot (h \cdot x) = x$.

				Act by $g$ on both sides. 
					$g\cdot x = g\cdot (\inv g \cdot (h \cdot x)) = (g\inv g)\cdot (h\cdot x) = h\cdot x$, so $g\approx h$.
			\item 
			Suppose $g\cdot h = h \cdot x$. Act by $\inv g$ on both sides. $\inv g \cdot (g\cdot h) = \inv g (h\cdot x)$.

			$x = e_G\cdot x = (\inv gh)\cdot x$, thus $\inv gh \in G_x$.
		\end{itemize}
		\qedhere(Claim 3)
	\end{proof}
	So the upshot is that we have a single equivalence relation on $G$. It has $|\orbit|$ -many classes. Each class has $|G_x|$-many elements. Hence $|G| = |\orbit|\cdot|G_x|$.\qedhere(Theorem~\ref{OST})
\end{proof}

\begin{remark}
	The above theorem also works when $G$ is not finite, however it involves multiplication of ordinals, which is beyond the scope of this course.
\end{remark}
\end{document}
