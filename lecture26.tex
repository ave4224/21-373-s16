\documentclass[notes.tex]{subfiles}

\newtcolorbox{fire}{colback=red!15,colbacktitle=orange!25,coltitle=black,boxrule=.5pt,parbox=false,oversize,title=Fire Alarm}

\begin{document}
\lecture{26}{2016--03--25}
\begin{classnote}{orange!10}{orange!25}
	Exam 2 will be on Monday April 11th

	(Also HW 9 is posted, don't forget they mention that $R$ has a $\one\ne\zero$)
\end{classnote}

\begin{defn}
	If $R$ is a commutative ring, $R[x] = $ polynomials with coefficients in $R$ = $\sum_{i=0}a_ix^i$. (Note that if $R$ has no \one then $x^0$ doesn't exists so just write it as $a_0 + \sum_{i=1}a_ix^i$).
\end{defn}

\begin{property}
	\[
		\left(\sum_{k=0}^ma_kx^k\right)
		\times\left(\sum_{k=0}^nb_kx^k\right)
		= \sum_{k=0}^{m+n}\left(\left(\sum_{i=0}^ka_ib_{k-i}\right)x^k\right)
	\]
\end{property}

Last time: we showed that given a commutative ring $R$ with a \one, and an ideal $I\subseteq R$.
\begin{itemize}
	\item $I$ is maximal $\iff R/I$ is a field.
	\item $I$ is prime $\iff R/I$ is an ID.
\end{itemize}
The first point generalizes $\Zn p$ is a field for prime $p$.
The ``other'' way to make a field out of $\ZZ$ is ``$\QQ$.''

\chapter{Fields of Fractions}
\begin{theorem}
	Suppose $R \ne \{0\}$ is a commutative ring
	with no zero-divisors. Then there is a field $Q$ with a subring $R'\subseteq Q$ with $R\cong R'$
\end{theorem}
\begin{proof}[Proof (sketchy)]
\footnote{Side note to the reader: if you've ever constructed the rationals from the integers before, we'll be following approximately that sort of procedure.}

Put $D := R \setminus\{0\}$. $D$ satisfies:
\begin{itemize}
	\item $D\ne \zero$ (because $R\ne\{\zero\}$)
	\item $\zero\notin D$
	\item $D$ contains no zero-divisors
	\item $\forall a, b\in D (ab\in D)$
\end{itemize}
$D$ will be the set of valid ``denominators'' in our construction.

Formally define $F = R\times D = \{(r, d) : r\in R, d\in D\}$.

Think of $(r, d)$ as ``$\frac{r}{d}$''.

Define $\approx$ on $F$ by $(r, d) \approx (s, e) \iff re = sd$.

Exercise: Prove that $\approx$ is an equivalence relation.

Denote by $Q$ the set $F/\approx$ of equivalence classes of $\approx$.

Write $\frac{r}{d}$ for the equivalence class $[(r, d)]_\approx$.

So, $\frac{r}{d} = \frac{s}{e} \iff (r, d)\approx(s, e) \iff re = sd$.

Formally ``define'' addition and multiplication in $Q$ by
\begin{align*}
	\frac{a}{b} + \frac{c}{d} &= \frac{ad+cb}{bd}
	& \frac{a}{b} \times \frac{c}{d} &= \frac{ac}{bd}
\end{align*}
Note that $bd\in D$ because $D$ is closed under multiplication.

\begin{claim}[1]
	Addition in $Q$ is well-defined
\end{claim}
\begin{proof}[Proof that addition is well defined (C1)]
	We need to show if $\frac{a}{b} = \frac{a'}{b'}, \frac{c}{d}= \frac{c'}{d'}$ (i.e., $ab' = a'b$ and $cd' = c'd$), then $\frac{a}{b} + \frac{c}{d} = \frac{a'}{b'}+\frac{c'}{d'}$. I.e., $\frac{ad+cb}{bd} = \frac{a'd'+c'b'}{b'd'}$, i.e., $(ad+cb)b'd' = (a'd'+c'b')bd$.

	\begin{align*}
		(ad+cb)b'd' &= adb'd' + cbb'd'\\
		&= (ab')dd' + (cd')bb'\\
		&= (a'b)dd' + (c'd)bb'\\
		&= a'd'bd + c'b'bd\\
		&=(a'd'+c'b')bd
	\end{align*}
\end{proof}
\begin{claim}[2]
	Multiplication in $Q$ is well-defined
\end{claim}
\begin{proof}[Proof that multiplication is well defined (C2)]
	The proof is left as an exercise to the reader.\qedhere(C2)
\end{proof}

\begin{claim}[3]
	$(Q, +, \times)$ is a ring
\end{claim}
\begin{proof}[Proof of Claim 3]
	$(Q, +)$ is an abelian group as we already know that $+$ is commutative and associative.

	\paragraph{Subclaim} $\forall d, d'\in D$, $\frac{\zero}{d} = \frac{\zero}{d'}$.
	\paragraph{Pf (subclaim)} $\zero d' = \zero = \zero d$\qed

	Let $\zero_Q$ denote the unique element of $Q$ of the form $\frac{\zero}{d}$ for $d\in D$.
	Check: $\frac{\zero}{d} + \frac{a}{b} = \frac{0b+ad}{db} \overop=?\frac{a}{b}$ Note that $adb = abd,$ so the final equality holds.

	Thus, $\frac{\zero}{d} = \zero_Q$ is the additive identity.

	We assert that $-\left(\frac{a}{b}\right) = \frac{-a}{b}$.

	Next, we show that $\times$ is an associative binary operator on $Q$ and that the distributive property holds.

	\begin{fire}
		Unfortunately at this point, the fire alarm went off. It is left to the reader to complete the proof or look up the remainder of it in the textbook.
	\end{fire}
\end{proof}

\qedhere(Thm)
\end{proof}

\end{document}
