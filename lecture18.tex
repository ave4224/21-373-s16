\documentclass[notes.tex]{subfiles}

\begin{document}
\lecture{18}{2016--02--24}

\begin{theorem}
	Suppose $p<q$ primes and $q \not\equiv 1\pmod p$. Then every group $G$ with $|G| = pq$ is cyclic. (i.e., $G\cong \Zn{pq}$)
\end{theorem}
\begin{proof}
	Let $p=3, q =5$, The general case proof is left as an exercise.

	Note: $\Zn p \times \Zn q$ ic cyclic because $|(h, k)| = \lcm (p, q) = pq$ (as they're relatively prime).

	So now there is enough to show that $G\cong \Zn p \times \Zn q$.
	By \nameref{cauchy}, we may take subgroups $H, K\le G$ with $|H| = p$ and $|K| = q$.
	We want: $\forall h\in H, \forall k\in K: hk = kh$ (then $G\cong \Zn p \times \Zn q$).

	First we let $K$ act on the subgroups of $G$ by conjugation. $k\cdot J = kJ\inv k$

	$|K| = 5$, so every orbit has size 1 or 5 by the \nameref{OST}. In particular, the orbit of $H$ $\orbit_H$ has size 1 or 5.

	\paragraph{Case 1} $|\orbit_H| = 1$. I.e., $\forall k\in K: kH\inv k = \{kh\inv k : h\in H\} = H$.

	Now let $K\actson H$ by conjugation. $k\cdot h = kh\inv k$. Orbits have size 1 or 5. But $|H| = 3$ So every orbit has size 1. I.e., $\forall k\in K, \forall h\in H: kh\inv k = h$, so $kh = hk$. Hence (in this case), $G\cong H\times K$.

	\paragraph{Case 2} $|\orbit_H| = 5$ I.e., there are 5 distinct subgroups of the form $kH\inv k, k\in K$. Then $G = \bigcup_{S\in \orbit_H}S \cup K$, as each pair of subgroups in $\orbit_H$ has intersection $e_G$. Also $K$ exists, so the subgroups of $G$ are either in $\orbit_H$ or equal to $K$.
	Thus, $K$ is the \emph{unique} subgroup of $G$ of cardinality 5. In particular, $\forall g\in G: gK\inv g$ also has cardinality 5, so $gk\inv g = K$, hence $K\nsubgrp G$.
	Let $H\actson K$ by conjugation. $h\cdot k = hk\inv h \in K$. 
	
	\begin{claim}[1]
		$\exists k_0\in K$ such that $k_0\ne e_G$ and $\forall h\in H: hk_0\inv h = k_0$
	\end{claim} 
	\begin{proof}
		Let $p$ be the number of fixed points. 
		By \nameref{FPL},$ p\equiv |K|\pmod 3$.
		However, $|K|\not\equiv 1\pmod 3$, so $p \ge 2$.
		Thus, there exists a non-identity fixed point ($k_0$)\footnote{Note that this is where we needed the hypothesis that $q\not\equiv 1\pmod p$.}.
		\qedhere(Claim)
	\end{proof}

	So, $\forall h\in H: hk_0 = k_0h$.
	Since $k_0\ne e$, $|k_0| = 5$, hence $K = \csg{k_0}$.
	Since $hk_0^n = k_0^nh$, and $h$ was arbitrary, we know that 
	$\forall h\in K, \forall k\in K: hk=kh$. Thus, $G\cong H\times K$.
	\qedhere(Thm).
\end{proof}

\begin{proposition}
	(p=2) Suppose $q > 2$ is prime. Then there exists (up to isomorphism) a \emph{unique} non-cyclic group $G$ with $|G| = 2q$.
\end{proposition}

\begin{proof}[Proof Sketch]
	Fix (by \nameref{cauchy}) $x, y\in G$ such that $|x| = 2$, $|y| = q$, then put $H = \csg{x}$ and $K = \csg y$.

	We posit that $K\nsubgrp G$ (verification left as an exercise).

	So $H\actson K$ by conjugation.
	So, $xy\inv x\in K$ $\exists r: 0 < r < q$ such that $xy\inv x = y^r$,
	$xy^2\inv x=(xy\inv x)(xy\inv x) = y^{2r}$. Thus, it is apparent that $xy^j\inv x = y^{jr\pmod q}$

	But also, \[
		y
		=x^2y^2x^{-2}
		=x(xy\inv x)\inv x
		=x(y^r)\inv x
		=y^{r\cdot r}
	\]
	So, $r^2\equiv 1\pmod q$

	Exercise: The only solutions to $r^2\equiv q\pmod(q)$ are $r\equiv 1, -1\pmod q$.

	\paragraph{Case 1} ($r=1$): $xy\inv x  = y$, i.e., $xy = yx$. Hence, $G\cong \csg x\times \csg y = H\times K$, cyclic.
	\paragraph{Case 2} ($r=-1$): $xy\inv x = \inv y$. Shuffle to obtain $yx = x\inv y$.

	We know that $G = \{x^iy^j : i\in \{0, 1\}, j\in \{0, \ldots, q-1\}\} = HK$.

	To multiply, $(x^iy^j)(x^\ell y^k) = \begin{cases}
		(x^iy^{j+k}) &\txtif \ell = 0\\
		x^i(y^jx)y^k = x^ixy^{-j}y^k = x^{i+1}y^{k-j} &\txtif \ell = 1
	\end{cases}$

	Hence, the group operation is completely determined by $yx=x\inv y$, and so there is at most one non-cyclic group (up to isomorphism) of cardinality $\Zn q$.

	To show the existance of such a group, given $n \ge 3$, we consider the \kw{dihedral group} of cardinality $2n$, where the dihedral group is the group generated by reflection and rotation on the regular $n$-polygon.
\end{proof}

\end{document}
