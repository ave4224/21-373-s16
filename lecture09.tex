\documentclass[notes.tex]{subfiles}

\begin{document}

\lecture{9}{2016--02--03}

\begin{classnote}{orange!10}{orange!25}
\emph{Midterm 1} is on Friday February 26th (in class)
\end{classnote}

Last time: showed that the kernel of a homomorphism is a subgroup.

\begin{proposition}
	$K\nsubgrp G$.
\end{proposition}

\begin{proof}
	First we show $K\le G$.
	\begin{itemize}
		\item $K \ne \vs$ as $e_G\in K$.
		\item $\forall g_1, g_2\in K, g_1g_2\in K$:
		\tabin
			$\vphi(g_1g_2) = \vphi(g_1)\vphi(g_2) =e_He_H = e_H$
			So $g_1g_2\in K$
		\tabout
		\item $\forall g\in H, \inv g \in K$:
		\tabin
			$\vphi(\inv g) = (\vphi(g))^{-1} = \inv e_H = e_H$
		\tabout
		So $\inv g \in K$
	\end{itemize}
	Thus, $K\le G$.
	Next, we prove.
	$\forall k\in K, \forall g\in G$:
		\[
			\vphi(\inv gkg)
			= \vphi(\inv g)\vphi(h)\vphi(g)
			= (\vphi(g))^{-1}e_H\vphi(g)
			= e_H
		\]
	Hence $\inv gkg\in K$.
\end{proof}

\begin{defn}
	If $\vphi :G\to H$ is a group homomorphism, and $h\in H,$ the \kw{fiber above $h$} is the set $\inv\vphi(\{h\})$.
\end{defn}

Thus, $\Ker(\vphi)$ is the fiber above $e_H$.

\begin{eg}\leavevmode
\begin{itemize}
	\item $\vp(\RR, +)\to(\RR^+, \times)$, $\vp(r) = \mathrm{e}^r$
	$\vp$ is a homomorphism since $\vp(r+s) = \mathrm{e}^{r+s} = \mathrm{e}^r\mathrm{e}^s = \vp(r)\times \vp(s)$

	$\Ker(\vp) = \{r\in R : \vp(r) = 1\} = \{0\}$.

	The fiber above $s\in \RR^+$:
		$\vp(r) = s \iff \mathrm{e}^r=s \iff r = \ln s$. Thus $\inv\vp(\{s\}) = \{\ln s\}$.

	\item $\vp:(\CC\setminus\{0\}, \times)\to(\RR\setminus\{0\}, \times)$,
		$\vp(a+b\iu) = a^2 + b^2$.

		$\vp$ is a homomorphism (verification left to the reader).

		$\Ker(\vp) = \{a+b\iu : \vp(a+b\iu) = 1\} = \{a+b\iu : a^2 + b^2 = 1\}$,
		which is the unit circle in the complex plane.

		Fix $r\in\RR\setminus\{0\}$, lets examine the fiber above $r$:
		\[
		\{a+b\iu : a^2 + b^2 = r\} =
		\begin{cases}
			\vs &\txtif r < 0\\
			\text{Circle of radius }\sqrt{r}&\txtif r > 0
		\end{cases}
		\]
	\item Start with a group $G$, normal $N\nsubgrp G$.
		$\vp:G\to G/N$, $\vp(g) = gN$ is a homomorphism.

		\begin{proof}
			$\vp(g_1g_2) = (g_1g_2)N = (g_1N)(g_2N) = \vp(g_1)\vp(g_2)$
		\end{proof}

		$\Ker(\vp) = \{g:\vp(g) = eN\} = \{g:\vp(g) = eN\} = N$.
\end{itemize}
\end{eg}

This leads us to the realization that:
\begin{proposition}
	$N\nsubgrp G \iff N = \Ker(\phi)$
	for some homomorphism $\vp:G\to H$, for any group $H$.
\end{proposition}

Why do all fibers look alike?

\begin{proposition}
	If $\phi:G\to H$ is a group homomorphism and $h\in H$,
	then $\inv\vp(\{h\})$ is either $\vs$ or $gK$ for some $g\in G$, where $K = \Ker(\vp)$
\end{proposition}

\begin{proof}
	If $\nexists g\in G$ such that $\vp(g) = h$ then $\inv\vp(\{h\}) = \vs$.

	Else, fix some $g\in G$ such that $\vp(g)=h$.
	\begin{claim}[1]
		$gK \supseteq \inv\vp(\{h\})$
	\end{claim}
	\begin{proof}
		Suppose $g' \in \inv\vp(\{h\})$ (i.e. $\vp(g')=h$). Want $g'\in gK$.
    So, $\vp(g \inv{g'}) = \vp(g)\vp(\inv{g'}) = \vp(g)\vp(g')^{-1} = h \inv h=e_H$.
		Hence $g \inv{g'} \in K$, so $g'\sim g \pmod K$, so $g'\in gK$.
		\qedhere(C1)
	\end{proof}
	\begin{claim}[2]
		$gK \subseteq \inv\vp(\{h\})$
	\end{claim}
	\begin{proof}
		Suppose $g' \in gK$, want $\vp(g')=h$.
		Fix $k\in K$ such that $g'=gk$.
		$\vp(g') = \vp(gk) = \vp(g)\vp(k) = he_H = h$.
		\qedhere(C2)
	\end{proof}
	Thus, $gK = \inv\vp(\{h\})$, as desired.\qedhere(Prop.)
\end{proof}

\begin{corollary}
	If $\vp:G\to H$ is a group homomorphism, the following are equal:
	\begin{itemize}
		\item $\vp$ is injective.
		\item $\Ker(\vp) = \{e_G\}$
	\end{itemize}
\end{corollary}

\begin{defn}
	A map $\vp:G\to H$ between groups is an \kw{isomorphism} if it is a bijective homomorphism.
	We often say $G\cong H$ if there exists an isomorphism $\phi:G\to H$.
\end{defn}

Intuition: Isomorphic groups have the ``same operation'' on different sets.

\begin{eg}
	Let $G = \{a, b\}$
	\begin{tabular}{r|cc}
		& $a$ & $b$\\\cline{2-3}
		$a$ & $a$ & $b$\\
		$b$ & $b$ & $a$
	\end{tabular}
	Then, $G\cong \ZZ/2\ZZ$ via $\vp : a\mapsto \bar 0, b\mapsto\bar1$
\end{eg}

We know $\vp:G\to H$ is an isomorphism if it's a homomorphism, surjective, and $\Ker(\vp) = \{e_G\}$.
\end{document}
