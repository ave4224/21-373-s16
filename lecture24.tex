\documentclass[notes.tex]{subfiles}

\begin{document}
\lecture{24}{2016--03--21}
First, we're going to look at the ring-theoretic equivalent of the first isomorphism theorem.

\begin{proposition}
	Suppose $R, S$ are rings, $\vp:R\to S$ is a (ring) homomorphism. Then $\im(\vp) = \{s\in S : \exists r\in R (\vp(r) = s)\}$
	is a subring of $S$.
\end{proposition}
\begin{proof}
	We already know that $(\vp[R], +) \le (S, +)$. So, $\vp[R]\ne\vs$, and it's closed under $+$ and $-$.

	Thus, we only need to check closure under $\times$.
	Fix $s_1, s_2\in \vp[R]$. Fix $r_1, r_2$ with $\vp(r_i) = s_i$. Want $s_1s_2\in \vp[R]$.

	Check $\vp(r_1r_2) = \vp(r_1)\vp(r_2) = s_1s_2$. Thus, $s_1s_2\in \vp[R]$.
\end{proof}

\begin{defn}
	A (ring) \kw{isomorphism} is a bijective (ring) homomorphism.
	If $\exists \vp:R\to S$ an isomorphism, then we write $R\cong S$.
\end{defn}

\begin{theorem}[First Isomorphism Theorem for Rings]
\label{FITR}
\addtoindex{First Isomorphism Theorem for Rings}.
Suppose $\vp: R\to S$ is a ring homomorphism. Then $R/\Ker(\vp)\cong \vp[R]$.
\end{theorem}
\begin{proof}
	Put $K:= \Ker(\vp)$
	By \nameref{FIT}, the map $\Psi:R/K\to \im(\vp)$ with $\Psi(r+K) = \vp(r)$ is a bijective group homomorphism from $(R/K, +)$ to $(\im(\vp), +)$.
	To verify that $\Psi$ is a ring isomorphism, we just need it to preserve $\times$.

	Fix $r_1+K, r_2+K \in R/K$. $\Psi((r_1+K)(r_2+K)) = \Psi(r_1r_2 + K) = \vp(r_1r_2) = \vp(r_1) \vp(r_2) = \Psi(r_1+K)\Psi(r_2 + K)$
\end{proof}

Ideals are the same thing as kernels of homormorphisms.

More on ideals:

\begin{proposition}
	If $I, J\subseteq R$ are ideals, then $I\cap J$ is an ideal.
\end{proposition}
\begin{proof}
	See Homework 7.
\end{proof}

\begin{proposition}
	If $I, J\subseteq R$ are ideals, then $I+J = \{i+j: i\in I, j\in J \}$ is an ideal.
\end{proposition}
\begin{proof}
	We need to check our ideal criteria.

	Non-zero: $\zero = \zero + \zero \in I+J$.

	Closed under $+$: Fix $i+j, i'+j' \in I+J$. Then $(i+j) + (i'+j') = (i+i') + (j+j') \in I+J$

	Closed under $-$: Fix $i+j\in I+J$. Then $-(i+j) = (-i)+(-j) \in I+J$

	Closed under $\times$ by $R$:
	Fix $i+j\in I+J$ and $r\in R$.
	Then $r(i+j) = ri + rj \in I+J$ and $(i+j)r = ir + jr \in I+J$
\end{proof}

\begin{defn}
	Fix an $a\in R$, $R$ a ring. Let $(a)$ denote the smallest ideal containing $a$.
	I.e., $(a)$ is the intersection of all ideals containing $a$.
	We call $(a)$ the \kw{principal ideal} generated by $a$.

  If $A\subseteq R$ we denote by $(A)$ the smallest ideal containing all elements of $A$. I.e., \[(A) = \bigcap_{\substack{I\text{ an ideal}\\I \supseteq A}}I.\]
\end{defn}
\begin{remark}
	$I+J = I\cup J$
\end{remark}

\begin{proposition}
	Suppose $R$ is a commutative ring with a \one.

	Then $(a) = aR = \{ar : r\in R\}$. $(a) = R$ iff $a$ is a unit.
\end{proposition}
\begin{proof}
	\begin{tabin}
		First show $aR \subseteq (a)$:
	\end{tabin}
	Fix $r\in R$. We know $a\in (a)$ So $ar\in (a)$. Hence $aR \subseteq (a)$.
	\begin{tabin}
		Next show $aR\supseteq (a)$:
	\end{tabin}
	Since $(a)$ is the ``smallest'' ideal containing $a$, it suffices to check that $aR$ is an ideal containing $a$.
	\begin{enumerate}
		\item ($a\in aR$): $a=a\cdot \one \in aR$ (used that $\one \in R$).
		\item ($aR$ is nonempty): $\zero = a\zero \in aR$
		\item (Closure under $+$): $ar_1 + ar_2 = a(r_1 + r_2) \in aR$
		\item (Closure under $-$): $-(ar) = a(-r) \in aR$
		\item (Closure under $\times$ by $R$): Fix $ar\in aR, s\in R$
		First $(ar)s = a(rs) \in aR$. Second, $s(ar) = a(sr) \in aR$ (uses commutativity).
	\end{enumerate}
\end{proof}
\begin{remark}
	If $R$ is not commutative but it still has a \one, we have if $a$ is a unit then $(a) = R$. Otherwise, $(a) = \{\sum_{i=1}^kr_ias_i : r_i, s_i\in R, k\in \ZZ^+\}$.
\end{remark}
\begin{remark}
	Warning: if $R$ has no \one, then typically $(a)$ is much bigger than $aR$ (or $RaR$).
	For example, $R = 5\ZZ, a=10$, $(a) = 10\ZZ$ but $aR = \{\ldots, -100, -50, 0, 50, 100, \ldots\}$.
\end{remark}

\begin{proposition}
	\label{prop:unitideal}
	Suppose $R$ has a \one. $I\subseteq R$ is an ideal. Then $I=R$ iff $I$ contains a unit.
	(So $(u) = R$ iff $u$ is a unit).
\end{proposition}
\begin{proof}
($\Longrightarrow$): Assume $I=R$. Then $\one \in I$.

($\Longleftarrow$): Suppose $u\in I$, $u$ a unit. $\exists v\in R$ such that $uv =\one$. $\one = uv \in I.$ Now fix $r\in R$. Then, $r = \one\cdot r\in I,$ so $I=R$ as desired.
\end{proof}

\begin{corollary}
	Suppose $R$ is a commutative ring with $\one\ne\zero$.
	Then $R$ is a field if its only ideals are $(\zero) = \{\zero\}$ and $R$.
\end{corollary}
\begin{proof}
	($\Longrightarrow$): Suppose $R$ is a field and $I\subseteq R$ is an ideal. If $I = \{\zero\}$ then we're done. Otherwise, $\exists r\ne \zero$ with $r\in I$. Since $r$ is a unit, $I = R$.

	($\Longleftarrow$): Suppose $R$ has only $\{\zero\}$ and $R$ for its ideals. Fix $\zero\ne r\in R$. Want to show that $r$ is a unit.
	Look at $(r)$. We know $r\in (r),$ so $r\ne \zero$.  Hence $(r) = R$. We also know that $(r) = rR$ by our assumptions along with proposition~\ref{prop:unitideal}. So, $\exists s\in R$ such that $rs = \one$. We also have $sr = \one$. Thus $r$ is a unit.
\end{proof}
\end{document}
